\documentclass[a4paper]{article}

\usepackage[english]{babel}
\usepackage[utf8]{inputenc}
\usepackage{amsmath}
\usepackage{graphicx}
\usepackage[colorinlistoftodos]{todonotes}

\title{LazyCoin}

\author{Sam Lavigne, Brian Clifton, Karl Ward, Jon Wasserman, Surya Mattu}

\date{\today}

\begin{document}
\maketitle

\abstract
A new currency that stores non-value.


\section{Introduction}

\begin{quote}
``Why do you refuse?''

``I would prefer not to.''

- Herman Melville
\end{quote}


LazyCoin is a currency that quantifies lack of activity. With LazyCoin, the more you do nothing the more value you create. 


\section{The Minting Process: Proof of Non-Work}

Minting new LazyCoin requires participation from at least two parties: one or more Generators, and one Verifier. The role of the Generator is to do nothing. The Verifier observes the Generator to ensure that he or she is doing nothing. When the Generator finishes producing LazyCoin, the Verifier \textit{signs} a blank LazyCoin, making note of the date and the amount of LazyCoin produced. The Verifier gives the newly minted, certified LazyCoin to the Generator to complete the minting process. 

The smallest unit of LazyCoin is minted in one minute.

During the minting process, the Verifier may choose to continue his or her normal, productive activity while periodically checking in on the Generator. The Verifier may also produce LazyCoin along with the other Generators. Should the Verifier choose the latter, he or she must receive Verification from the other Generators.

Proper verification requires that all parties remain in line of sight for the duration of the minting process.

The Generator should never do anything while minting LazyCoin, with the exception of the normal human metabolic process. However, under certain circumstances, and with prior approval by a Verifier, the Generator may pretend to do something.


\section{Currency Cap}
The daily currency cap of LazyCoin is limited by the number of minutes in a day, and the number of people on the planet. According to current population estimates, this means that the maximum possible LazyCoins that can be produced in one day is 10,303,200,000,000.


\section{Verification Issues and Resolutions: A System of Trust}
The extremely young, the extremely old, the incarcerated, the homeless, and others excluded from the productive workforce are the most natural producers of LazyCoin.

However, the extremely young are not mentally developed enough to be each others’ Verifiers, and the extremely old have higher risk of conditions that could compromise their ability to effectively Verify (such as senility or other diseases that affect the mind). 

To protect against potential exploitation, LazyCoin can only be minted by those who are aware they are minting LazyCoin. 

LazyCoin is not immune to forgery. Instead it relies on a system of trust and reputation: if a Verifier is caught forging LazyCoin his or her reputation in the LazyCoin community will decrease. Verifiers who fraudulently sign LazyCoins, or sign LazyCoins minted by “zombie” Generators will inevitably be exposed and removed from the system.

\section{Double-Earning}
One benefit of LazyCoin is the possibility of minting LazyCoin while simultaneously earning standard currency. For example, a LazyCoin Generator may choose to mint LazyCoin while working a salaried job, simply by ceasing normal work activities and doing nothing. If the Generator earns income through freelance or hourly work, he or she can attempt to simultaneously mint LazyCoin by padding hours or making fraudulent entries on a timecard. 

\section{Exchanging LazyCoin}
As with all currencies, LazyCoin can only be exchanged with those who are willing to exchange it. During the initial phase of LazyCoin minting we recommend that all LazyCoin Generators accumulate as much LazyCoin as possible and store their earnings under mattresses and floorboards for later non-use.

Unlike typical ``fiat'' currencies that require mutual participation from multiple parties for transactions to take place, LazyCoin can be exchanged with consent from only one party: the person holding LazyCoin. For example, a person holding LazyCoin could enter a store and deliver LazyCoin in exchange for buying nothing.


\end{document}
